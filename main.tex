\documentclass[12pt,a4paper]{article}
\usepackage[utf8]{inputenc}
\usepackage[T1]{fontenc}
\usepackage[french]{babel}
\usepackage{graphicx}
\usepackage{geometry}
\usepackage{setspace}
\usepackage{titlesec}
\usepackage{xcolor}
\usepackage{lmodern} % Police plus moderne et lisible

% Mise en page
\geometry{margin=2.5cm}
\setstretch{1.2}

% Style des titres
\titleformat{\section}
  {\large\bfseries\color{blue!70!black}}
  {\thesection}{1em}{}

\titleformat{\subsection}
  {\normalsize\bfseries\color{blue!50!black}}
  {\thesubsection}{1em}{}

% Alinéa
\setlength{\parindent}{15pt}
\setlength{\parskip}{6pt}

\begin{document}

\begin{center}
    {\LARGE \textbf{Analyse d'une visualisation -- Temps d'écran}}\\[0.5cm]
    \textbf{Thiebaud Enzo} \\
    Numéro étudiant : \textbf{p2207446} \\[0.5cm]
\end{center}

\section*{Description des données}

La visualisation provient de l'étude \textit{Esteban 2015}.
Elle présente la durée quotidienne moyenne passée devant un écran
(télévision, ordinateur ou console de jeux) chez les enfants âgés de 6 à 17 ans.
Les données sont distinguées selon le sexe (garçons et filles)
et selon le niveau de diplôme de la personne de référence du ménage.

\begin{figure}[h!]
    \centering
    \includegraphics[width=0.9\textwidth]{ecran1.jpg}
    \caption{\small Durée quotidienne moyenne passée devant un écran selon le sexe et le diplôme, étude Esteban 2015.}
\end{figure}

\section*{Encodage visuel}

L'encodage principal repose sur la hauteur des barres,
qui indique le nombre moyen d'heures quotidiennes passées devant un écran.
Les barres bleues représentent les garçons et les barres rouges les filles.
L'axe horizontal présente les catégories de diplômes,
tandis que les barres d'erreur montrent la variabilité des données.
L'échelle verticale (de 0 à 6 heures) facilite la comparaison entre les groupes.

\section*{Interactions}

La visualisation est entièrement statique, aucune interaction n'est possible.
Elle reste néanmoins claire et permet une lecture rapide des tendances générales.

\section*{Analyse et critique}

On observe que le temps d'écran diminue lorsque le niveau de diplôme du ménage augmente.
Les garçons passent en moyenne plus de temps devant les écrans que les filles,
quelle que soit la catégorie de diplôme.

Cependant, certaines incohérences sont visibles :
\begin{itemize}
    \item Le titre indique que les données concernent les enfants de 6 à 17 ans,
          mais les diplômes mentionnés (\textit{Bac+3}, \textit{Bac+4}, etc.) ne peuvent pas s'appliquer à des enfants.
          Cela montre que le diplôme concerne en réalité les parents, ce qui n'est pas précisé clairement dans le titre.
    \item L'absence de légende explicite pour les couleurs peut créer une légère confusion
          pour un lecteur non averti.
    \item Le titre du graphique pourrait être reformulé pour indiquer plus clairement qu'il s'agit
          du lien entre le niveau d'éducation du foyer et le temps d'écran des enfants.
\end{itemize}

Malgré ces remarques, la visualisation reste efficace pour montrer la tendance générale
et illustre bien la relation entre le niveau socio-éducatif et le temps d'écran des jeunes.

\end{document}
